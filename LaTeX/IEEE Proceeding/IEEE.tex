\documentclass[conference]{IEEEtran}
\IEEEoverridecommandlockouts
% The preceding line is only needed to identify funding in the first footnote. If that is unneeded, please comment it out.
\usepackage{cite}
\usepackage{amsmath,amssymb,amsfonts}
\usepackage{algorithmic}
\usepackage{graphicx}
\usepackage{textcomp}
\usepackage{xcolor}
\usepackage{float}
\def\BibTeX{{\rm B\kern-.05em{\sc i\kern-.025em b}\kern-.08em
    T\kern-.1667em\lower.7ex\hbox{E}\kern-.125emX}}
\begin{document}

\title{How Can Artificial Intelligence Help With Space
Missions - A Case Study: Computational
Intelligence-Assisted Design of Space Tether for
Payload Orbital Transfer Under Uncertainties*\\
{\footnotesize \textsuperscript{*}This work was supported in part by the National Natural Science Foundation of China under Grant 51875090, and in part by the Natural
Science Foundation of Guangdong Province under Grant 2018A030313320.
should not be used}
}

\author{\IEEEauthorblockN{1\textsuperscript{st} XIANLIN REN 1}
\IEEEauthorblockA{\textit{Member, IEEE} \\
}
\and
\IEEEauthorblockN{2\textsuperscript{nd} YI CHEN}
\IEEEauthorblockA{\textit{Senior Member, IEEE} \\
}

}

\maketitle

\begin{abstract}
In the era of artificial intelligence (AI), many industry sectors, including space exploration,
have experienced a shift in the way business is conducted due to the widespread use of AI technologies.
In the past few years, AI has become a key tool used to explore the universe in space missions. In this paper,
a multi-objective optimal design for payload orbital transfer involving space tethers is proposed based on
a computational intelligence-assisted design framework with the artificial wolf pack algorithm (AWPA).
Enlightened by the social behaviors of a wolf pack and its swarm intelligence, the AWPA is utilized for
optimization problems in which a logsig function randomly obtains assignments for parents and offspring.
Swarmwolf , a simulation toolbox with given initial conditions. The proposed method effectively performs
optimization tasks based on index of evolutionary pathway trends, has been defined to demonstrate the
optimizing process. The results show that the proposed approach works expeditiously for the optimization
of space tether model and its application.
\end{abstract}

\begin{IEEEkeywords}
Artificial intelligence, artificial wolf-pack algorithm, computational intelligence assisted
design, evolutionary pathway, multi-objective optimization, payload orbital transfer, space tether.
\end{IEEEkeywords}

\section{Introduction}
Reflecting one of the most innovative activities, artificial
intelligence (AI) has become essential to the global economy,
and its positive effects on society in the context of efficiency
are immeasurable and emerging in our daily lives. According
to an analysis by PwC, by 2030, AI could add £232BN to
the Gross Domestic Product of the United Kingdom, which
is approximately 10% higher than the current level. In the
past few years, AI has become a key tool in space activities
and the exploration of the universe, and it can be utilized
for space communications, navigation, big data analysis,
autonomy evaluation, decision support for spacecraft system design [3], space mission operations [4], [5], planetary
defense, mapping the moon [6], space exploration [7] and
The associate editor coordinating the review of this manuscript and
approving it for publication was Zhonglai Wang .
universe exploration [8], among other tasks. Most previous
works focused on space structure design, communications or
flight simulations, and few have addressed the ‘space tether’.
The original space tether concept came from the ‘space
elevator’, which was proposed by a Russian scientist in 1895.
A space tether is a type of lengthy cable ranging from a
few hundred meters to many kilometers. The cable makes
use of a few bundles of thin strands of high-strength fiber
to couple spacecraft to every different or other masses, and
it provides a mechanical connection that allows the transfer
of energy and momentum from one object to the other. Space
tethers can be utilised in many applications, collectively with
studies of plasma physics and electrical science in the Earth’s
upper atmosphere, the orbiting or deorbiting of space vehicles, payload transfer, inter-planetary propulsion, and specialised missions, such as asteroid engagement or, in severe
form, as a well-publicized space elevator development of space technology, space tethers are broadly
used in the exploration of Mars, or even deep space missions [2]. Recently, a few companies, e.g., Google and
SpaceX, have been working on space projects related to
space elevators, and these projects are mainly supported by
developed AI tools.
To fill the current research gap, this paper utilizes an
AI-driven design tool for payload orbital transfer via a
space tether using a computational intelligence-assisted
design (CIAD) framework [9], [10], [22], [28]. An intelligent
design system is created with the artificial wolf pack algorithm (AWPA) [24], and this gadget presents three practical
advantages: (1)the mobilisation and flexibility of computing
resources; (2) the embedment of a set of AI algorithms,
and (3) the reducing computational cost as a design objective. In this research, a symmetrical motorized momentum space tether (MMET) with a payload at each end
experiences centripetal acceleration when rotating about a
facility [1], [11], [12].
In the last a few decades, the studies on the nature-inspired
AI algorithms have been widely performed, such as the family of genetic algorithms (GAs) [25], which were inspired
by the Darwinian theory and the concept of survival of the
fittest; swarm intelligence and algorithms, including the ant
colony optimization in 1992, the particle swarm optimization (PSO) in 1995, the Artificial bee colony in 1996, and
the firefly swarm algorithm (Firefly) in 2013 [27], and others, based on the principles of natural phenomena, which
reproduce the social intelligence of the behaviour of an
animal or insect swarm. As one of the most broadly used
evolutionary algorithms, the GAs provide a near-optimal
solution for practical problems with massive numbers of
variables and constraints, in which, the best possible control
parameters, including the size of population, the crossover
operation rate and the mutation operation rate, are difficult to determine. The typical motivation behind developing nature-based algorithms is to effectively and efficiently
solve different optimization problems. During Earth’s natural
development, it is assumed that the behavior of nature is
usually optimal. In this research, the AWPA, a newly proposed nature-inspired algorithm, is utilised to achieve global
solutions for space tether’s continuous nonlinear functions in
a space mission, with low computational effort and excessive
consistency.
The remaining sections of this work are organized as
follows. Section II introduces the social dynamic behaviors of a WP in the natural world, then it gives the AWPA
algorithm with logsig randomness, which was stimulated by
using the WP’s swarm intelligence of its social behaviors,
as added in Section II. To investigate the optimization performance, Section III defines the fashion indices for evolutionary optimization, which includes three pairs of trend indices.
Section V offers the modeling of payload transfer the use
of a house tether. Section VI discusses multi-objective optimization and fitness functions based on the relevant criteria.
Section VII presents the empirical outcomes and a discussion 

\begin{figure}[h]
\centering
\begin{minipage}{0.4\textwidth}
    \includegraphics[width=1\textwidth]{A-WPs-social-behaviors.png}
    \caption{A WP's social behaviors}
\end{minipage}
\end{figure}

\section{WOLF PACK’S SOCIAL MECHANISM AND ARTIFICIAL
WOLF PACK ALGORITHM}

Wolves are gregarious animals who mostly live in packs.
A wolf pack (WP) is formed when a male ($\alpha$) and a female
wolf ($\beta$) meet and continue to be together as a mated pair.
The pair establishes a territory to settle and increase cubs in
most years. The cubs stay within the WP until they are old
enough to leave home, generally when they are 3 years old,
and they can then start a WP of their own. Thus, the social
structure of a WP consists of a permanent core of a $\alpha + \beta$
pair plus their continuously dispersing offspring. A WP has a
very strict stage of hierarchy that need to be adhered to by all
the members of the pack, and it supports the WP to survive.
For their social intelligence, wolves are considered one of
the smartest animals in nature, and they have developed the
potential to live on in a vast vary of surroundings, which
varies from the wilderness to humid swamps to the Arctic
region with temperatures of $-40$ and bitter winds.
Basically, a WP has 3 types of behaviors in their social
activities, namely, ‘scouting’, ‘calling’ and ‘besieging’,
as shown in Figure 1, usually in a WP, an $\alpha$ male wolf
is the leader wolf who is in the decision-making position

\begin{figure}[H]
\centering
\begin{minipage}{0.4\textwidth}
    \includegraphics[width=1\textwidth]{The-workflow-of-artificial-wolf-pack-algorithm.png}
    \caption{The workflow of ar tificial wolf pack algorithm.}
\end{minipage}
\end{figure}

and is able to command all the other wolves to carry out
suitable actions. Specifically, scouting: the scout group of
wolves are deployed to explore unknown areas; calling: when
the scout wolves have spotted and positioned the prey in an
area, they will report this information to the $\alpha$ wolf and also
communicate with other wolves by their howling; besieging:
after the $\alpha$ wolf confirms the prey information and will be
leading the WP toward the scout wolves, starting the grey
hunting and capturing actions in the so-called the besieging
area.
Inspired by the swarm intelligence of a WP’s social behaviors, an artificial intelligent algorithm, the Artificial Wolf
Pack Algorithm (AWPA) is proposed by implementing three
social behaviors of a WP, in which, there are six steps include:
(1) initialization of all the variables; (2) the behavior selection
from three social behaviors with the logsig() randomness;
(3) the operation of scouting behavior; (4) the operation of
calling behavior; (5) the operation of besieging behavior and
(6) the bulletin step to summarise the results.
Without loss of generality, in this section, Sj is considered the status of any artificial wolf with status j.
The AWPA workflow is given in Figure 2. As shown in
Equation (1), when performing the AWPA with a pre-defined
function WPM×N , it first simulates the behavior of an individual artificial wolf (AW, as Sij), where M is the population and
N is the number of individuals in a WP. Each AW searches
for the local optimal solution and passes its result to the self-organized WP and so as to obtain the optimal global
solution.
As also shown in Figure 3, it gives the pseudocode implementation of the AWPA algorithm, in which, the ‘maxgeneration’ parameter is set as the terminal condition of the
AWPA optimisation.

\begin{figure}[H]
\centering
\begin{minipage}{0.4\textwidth}
    \includegraphics[width=1\textwidth]{AWPA-Pseudo-Code_Q320.jpg}
    \caption{AWPA Pseudo Code}
\end{minipage}
\end{figure}

(1) Initialization: in this step, all the variables and parameters are set to the pre-defined values, in a case, population =
50, max generation = 200, etc., and the simulation program
prepare for the subsequent steps

\begin{align}
WP_{MXN} = \begin{bmatrix} 
S_{11} & S_{12} & S_{13} \ldots\ &S_{1N} \\
S_{21} & S_{22} & S_{23} \ldots\ &S_{2N}\\
S_{31} & S_{32} & S_{33} \ldots\ &S_{3N} \\
S_{M1} & S_{M2} & S_{M3} \ldots\ &S_{MN}
\end{bmatrix}
\quad
\end{align}\\
(2) Behavior Selection: In the ‘behavior selection’, there
are three types of ‘states’ to indicate three different types of
behaviors of a WP, that is, ‘scouting’, ‘calling’ and ‘besieging’, in which, the ‘scouting’ state is set as the default state
or initial behavior of each WP.
Depending on the companion’s number and the visual conditions, the prey density in the hunting region can be defined
in Equation (2).\\

\begin{align} 
visual(t) = logsig\left(\frac{\frac{T}{2}- t}{k}\right) x random(t)
\end{align}
in which, 

\begin{itemize}
    \item logsig() is a logarithmic sigmoid transfer function;
    \item T is the maximum number of iterations;
    \item t is the current iteration number;
    \item k changes the slope of the logsig() function;
    \item random() is a random value within the range of (0,1).
\end{itemize}

\begin{figure}[H]
\centering
\begin{minipage}{0.4\textwidth}
    \includegraphics[width=1\textwidth]{The-Euclidean-distance-between-the-AWi-th-and-AWj-th-and-states-updating.png}
    \caption{The Euclidean distance between the $AWi^{th}$ and $AWj^{th}$ and states updating}
\end{minipage}
\end{figure}

(3) Scouting: For an AW individual k in a WP, let's define
S as the finite state set, and there are states 1 to M that an AW
can perform, as given in Equation (3).
Within the AW's visual field, let's define $S_{(*)i}$ as the cur-rent state of an AW and $S_{(*)i}$ as the next state. Specifically,
an AW moves from its current state $S_{(*)i}$ to the next state $S_{(*)i}$
randomly, and keep checking the state updating conditions,
as stated in Equations (5) and (4), where $\epsilon$ is a random
movement factor, $\delta$ is the iteration step, and $\nu$ is the visual
constant of the AW. For a given AW, the prey density is
defined as z D f (S), in which, z is the fitness function,
z i and z j are the prey density in the state $S_{(*)i}$ and $S_{(*)i}$,
respectively.


\section{THREE PAIRS OF TREND INDICES}

To evaluate the performance of the optimization process, and
benchmarks for the AWPA algorithm, three pairs of trend
indices are introduced in this section:
1) the trend index of moving mean of the average pre-cision (mmAP) and the index of moving mean of the
standard deviation (mmSTD), as given in equations (13)
and (14), respectively;
The trend index of mmAP is dened as a moving aver-age score of the MEAN value of a vector fj, as given in
equation (13), in which $i = 1, 2, \ldots j,\ldots p$, p is the
size of the dataset's population, MEAN is the average
function.
The trend index of mmSTD is a moving average score
of the STD value of vector fj, as given in equation (14),
in which STD is the standard deviation function.
Both indices of mmAP and mmSTD are employed
to assess the short-term fluctuations by recording the
long-term trend throughout their evolutionary process.
2) the trend index of `moving max of the average pre-cision' (mmaxAP) and the index of `moving max of
the standard deviation' (mmaxSTD), as given in equa-tions (15) and (16), respectively.

\section{EVOLUTIONARY PATHWAY}
As shown in Figure 5, from top to bottom, the solid lines
are the trend indices of mmaxAP, mmAP and mminAP of a
fitness function in a vector fj, as given in equations (15), (13)
and (17), respectively.
The dashed lines are the boundaries of $mmaxAP +
mmaxSTD$, $mmAP + mmSTD$ and $mminAP + mminSTD$
values for each vector fj, as defined in Equations (16), (14)
and (18), which illustrate the tracking shape of the evolution-ary pathway for the optimization process, generation versus
fitness f in this case, as the upper and lower boundaries.

\section{PAYLOAD TRANSFER USING SPACE TETHER UNDER
UNCERTAINTIES}
As illustrated in Figure 6, there are two sub-figures labelled
as (a) and (b), which describes the orbital elements of payload
release and transfer using space tether.

\begin{figure}[H]
\centering
\begin{minipage}{0.4\textwidth}
    \includegraphics[width=1\textwidth]{Orbital-elements-of-payload-release-and-transfer_Q320.jpg}
    \caption{Orbital elements of payload release and transfer}
\end{minipage}
\end{figure}
\begin{itemize}
    \item Figure 6 (a), it demonstrates the payload transfer from a
3D perspective.
    \item Figure 6 (b), it describes the parameters for modeling of
the payload transfer from a 2D perspective.
\end{itemize}
Generally, the space tether's payload transfer is in a low
Earth orbit, where the space tether obtains rising angular
velocity than it requires to remain in the orbit, however, it does
not comprise sufficient energy to escape from the Earth's
gravity field.
During payload transferring, a robotic space tether, con-sists of a few modules including: the upper payload, the lower
payload and a center of mass (COM), is in a low Earth
orbit, where the upper payload can be released from the
spinning space tether with enough angular momentum, which
is aligned along the local gravity vector.
After the release operation, the upper payload will enter an
elliptical orbit (named as `OU'), as shown in both sub-figures
(a) and (b), in which the point A (the upper perigee of `OU')
is the release point. Then, half of moving on the elliptical orbit
later, the upper payload reaches the point B (the upper apogee
of `OU'), where is further from the Earth than it was at release
point.
On the contrary, the lower payload will not have enough
energy to maintain in its circular orbit after the release oper-ation, and it will enter into another elliptical orbit (named
as `LU'). Upon reaching the point D (the perigee of `LU'),
the lower payload is closer to the Earth than it was at the
release point C (the apogee of `LU').
Finishing above actions, the upper and lower masses of
payloads are released from the spinning space tether, entering
a raised orbit and lowered orbit, respectively.
All the environmental effects that can influence the
space tether modelling are assumed to be negligible in the
space tether's modelling context, including solar radiation,
aerodynamic drag, electrodynamic forces and no frictional
losses.
To build a model for the payload transfer, a set of parame-ters are defined here,

\begin{figure}[H]
\centering
\begin{minipage}{0.4\textwidth}
    \includegraphics[width=1\textwidth]{Optimisation-of-Space-Tether-for-Payload-Orbital-Transfer-via-CIAD-Framework.png}
    \caption{Optimisation of Space Tether for Payload Orbital Transfer via CIAD Framework}
\end{minipage}
\end{figure}

\section{FITNESS FUNCTIONS AND OPTIMAL DESIGN
CRITERIA}
As shown in Figure 7, the multi-objective optimization pro-cess of the space tether for payload orbital transfer includes
four steps, as listed below.
\begin{enumerate}
    \item Pre-processing: This step includes the dynamic mod-eling of the space tether for payload orbital transfer as given in Section V;
    \item The optimal design given by the computational intelli-gence approaches via the integrated solver;
    \item Three design objectives. The MMET payload transfer
and design strength problems are defined as three conflicting objectives that need to be balanced in the opti-mization study, in which the practical multi-objective
problem of tether payload transfer can be expressed as
follows [20]
    
\end{enumerate}

\begin{figure}[H]
\centering
\begin{minipage}{0.4\textwidth}
    \includegraphics[width=1\textwidth]{FPR Flowchar.png}
    \caption{FPR Flowchart}
\end{minipage}
\end{figure}

\section{OPTIMAL DESIGN FOR ORBITAL PAYLOAD
TRANSFER}
The computers used for the simulations have Intel Core
i7-5500U 2.4 GHz dual-core processors with the Win-dows 7 flagship x64 service pack 1, 8.0-GB 1600-MHz
dual-channel DDR3L SDRAM, and MATLAB R2010a.
Table 1 lists the parameters for the optimization
process, in which, the termination condition is set to`
max-generation D100' for each loop of evolutionary opti-misation; the try variable is 5; the population is 50; the total
number of tests is 100; the iteration step is 0.001; the initial
visual value is 2.5; the crowd value is 0.618.
Table 2 gives the optimal combinations (MEAN STD) of
z and the design variables (x1;x2;x3), and the results indicate
that the overall performance of the AWPA in the optimal design of a space tether for orbital payload transfer is better
than that of the algorithm and GA.
Figures 9 to 12 give the max, mean and min fitness curves,
which show that the AWPA converge near generation 30.
Figure 9, Figure 10 and Figure 11 show the mmaxAP
+ mmaxSTD, mmAP + mmSTD and mminAP + mminSTD
fitness curves over the full simulation period, respectively.
Figure 12 illustrates the evolutionary pathway of the opti-mization process of the optimal design for space tether pay-load transfer using the mmaxAP + mmaxSTD, mmAP +
mmSTD and mminAP + mminSTD fitness curves with their
upper and lower boundaries.
Obviously, the fitness values increases very quickly, and
it reaches a plateau from generations 1 to 250. All curves
converge from generation 280 to 300, which reflects the
high efficiency of the proposed AWPA for this space tether
payload transfer application.

\begin{figure}[H]
\centering
\begin{minipage}{0.4\textwidth}
    \includegraphics[width=1\textwidth]{Evolutionary pathway of space tether payload transfer.png}
    \caption{Evolutionary pathway of space tether payload transfer}
\end{minipage}
\end{figure}

\section{CONCLUSIONS AND FUTURE WORKS}
In this study, the efficiency of payload transfer considering
the perigee altitude loss  and tether stress  are defined for
space tether payload orbital transfer, and three fitness func-tions are obtained using the newly devised trend indices for the multi-objective problem. Then, optimization is performed
with the newly developed multi-objective AWPA.
The following conclusions can be drawn with the results
and discussions above

\begin{itemize}
    \item The optimal design using the multi-objective AWPA
exhibits good agreement with previous results.
    \item The evolutionary trends are accurately rejected by the
three defined pairs of trend indices with variable uncer-tainty and tolerance levels
    \item the dynamic evolutionary behaviors of three pairs of
trend indices are obtained and considered in optimiza-tion.
    \item the fast Pareto-optimal solution recommendation
method is introduced, and a clear recommendation list
is established for decision making.
    \item further studies on the implementation of experimental
verification and validation for the space tether and space
robotic and autonomous system;
    \item the development of the operational strategy of high effi-cient payload transfer using space tether systems;
    \item the studies on the high efficiency and reliability design
for the key components of the space tethers;
    \item to develop a computational intelligence assisted design
software package for industrial design and manufac-ture, as one of the components of cyber-physical sys-tem (CPS) for industry 4.0 applications in future space
projects, using cutting-edge technologies, such as: `dig-ital twin' and internet of things;

\end{itemize}


\end{document}
